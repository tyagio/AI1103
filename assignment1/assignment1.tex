\documentclass[journal,12pt,twocolumn]{IEEEtran}

\usepackage{setspace}
\usepackage{gensymb}
\singlespacing
\usepackage[cmex10]{amsmath}

\usepackage{amsthm}

\usepackage{mathrsfs}
\usepackage{txfonts}
\usepackage{stfloats}
\usepackage{bm}
\usepackage{cite}
\usepackage{cases}
\usepackage{subfig}

\usepackage{longtable}
\usepackage{multirow}

\usepackage{enumitem}
\usepackage{mathtools}
\usepackage{steinmetz}
\usepackage{tikz}
\usepackage{circuitikz}
\usepackage{verbatim}
\usepackage{tfrupee}
\usepackage[breaklinks=true]{hyperref}
\usepackage{graphicx}
\usepackage{tkz-euclide}

\usetikzlibrary{calc,math}
\usepackage{listings}
    \usepackage{color}                                            %%
    \usepackage{array}                                            %%
    \usepackage{longtable}                                        %%
    \usepackage{calc}                                             %%
    \usepackage{multirow}                                         %%
    \usepackage{hhline}                                           %%
    \usepackage{ifthen}                                           %%
    \usepackage{lscape}     
\usepackage{multicol}
\usepackage{chngcntr}

\DeclareMathOperator*{\Res}{Res}

\renewcommand\thesection{\arabic{section}}
\renewcommand\thesubsection{\thesection.\arabic{subsection}}
\renewcommand\thesubsubsection{\thesubsection.\arabic{subsubsection}}

\renewcommand\thesectiondis{\arabic{section}}
\renewcommand\thesubsectiondis{\thesectiondis.\arabic{subsection}}
\renewcommand\thesubsubsectiondis{\thesubsectiondis.\arabic{subsubsection}}


\hyphenation{op-tical net-works semi-conduc-tor}
\def\inputGnumericTable{}                                 %%

\lstset{
%language=C,
frame=single, 
breaklines=true,
columns=fullflexible
}
\begin{document}


\newtheorem{theorem}{Theorem}[section]
\newtheorem{problem}{Problem}
\newtheorem{proposition}{Proposition}[section]
\newtheorem{lemma}{Lemma}[section]
\newtheorem{corollary}[theorem]{Corollary}
\newtheorem{example}{Example}[section]
\newtheorem{definition}[problem]{Definition}

\newcommand{\BEQA}{\begin{eqnarray}}
\newcommand{\EEQA}{\end{eqnarray}}
\newcommand{\define}{\stackrel{\triangle}{=}}
\bibliographystyle{IEEEtran}
\raggedbottom
\setlength{\parindent}{0pt}
\providecommand{\mbf}{\mathbf}
\providecommand{\pr}[1]{\ensuremath{\Pr\left(#1\right)}}
\providecommand{\qfunc}[1]{\ensuremath{Q\left(#1\right)}}
\providecommand{\sbrak}[1]{\ensuremath{{}\left[#1\right]}}
\providecommand{\lsbrak}[1]{\ensuremath{{}\left[#1\right.}}
\providecommand{\rsbrak}[1]{\ensuremath{{}\left.#1\right]}}
\providecommand{\brak}[1]{\ensuremath{\left(#1\right)}}
\providecommand{\lbrak}[1]{\ensuremath{\left(#1\right.}}
\providecommand{\rbrak}[1]{\ensuremath{\left.#1\right)}}
\providecommand{\cbrak}[1]{\ensuremath{\left\{#1\right\}}}
\providecommand{\lcbrak}[1]{\ensuremath{\left\{#1\right.}}
\providecommand{\rcbrak}[1]{\ensuremath{\left.#1\right\}}}
\theoremstyle{remark}
\newtheorem{rem}{Remark}
\newcommand{\sgn}{\mathop{\mathrm{sgn}}}
\providecommand{\abs}[1]{$\left\vert#1\right\vert$}
\providecommand{\res}[1]{\Res\displaylimits_{#1}} 
\providecommand{\norm}[1]{$\left\lVert#1\right\rVert$}
%\providecommand{\norm}[1]{\lVert#1\rVert}
\providecommand{\mtx}[1]{\mathbf{#1}}
\providecommand{\mean}[1]{$\left[ #1 \right]$}

\providecommand{\fourier}{\overset{\mathcal{F}}{ \rightleftharpoons}}
%\providecommand{\hilbert}{\overset{\mathcal{H}}{ \rightleftharpoons}}
\providecommand{\system}{\overset{\mathcal{H}}{ \longleftrightarrow}}
	%\newcommand{\solution}[2]{\textbf{Solution:}{#1}}
\newcommand{\solution}{\noindent \textbf{Solution: }}
\newcommand{\cosec}{\,\text{cosec}\,}
\providecommand{\dec}[2]{\ensuremath{\overset{#1}{\underset{#2}{\gtrless}}}}
\newcommand{\myvec}[1]{\ensuremath{\begin{pmatrix}#1\end{pmatrix}}}
\newcommand{\mydet}[1]{\ensuremath{\begin{vmatrix}#1\end{vmatrix}}}
\numberwithin{equation}{subsection}
\makeatletter
\@addtoreset{figure}{problem}
\makeatother
\let\StandardTheFigure\thefigure
\let\vec\mathbf
\renewcommand{\thefigure}{\theproblem}
\def\putbox#1#2#3{\makebox[0in][l]{\makebox[#1][l]{}\raisebox{\baselineskip}[0in][0in]{\raisebox{#2}[0in][0in]{#3}}}}
     \def\rightbox#1{\makebox[0in][r]{#1}}
     \def\centbox#1{\makebox[0in]{#1}}
     \def\topbox#1{\raisebox{-\baselineskip}[0in][0in]{#1}}
     \def\midbox#1{\raisebox{-0.5\baselineskip}[0in][0in]{#1}}
\vspace{3cm}
\title{Assignment 1}
\author{Ojjas Tyagi - CS20BTECH11060}
\maketitle
\newpage
\bigskip
\renewcommand{\thefigure}{\theenumi}
\renewcommand{\thetable}{\theenumi}
Download all python codes from 
\begin{lstlisting}
https://github.com/tyagio/AI1103/tree/main/assignment1/codes
\end{lstlisting}
%
and latex-tikz codes from 
%
\begin{lstlisting}
https://github.com/tyagio/AI1103/tree/main/assignment1/assignment1.tex
\end{lstlisting}
\section{Problem}
Suppose that two cards are drawn at random from a deck of cards.Let X be the number of aces obtained.Then the value of E(X) is 
\begin{enumerate}
    \item 37/221
    \item 5/13
    \item 1/13
    \item 2/13
\end{enumerate}

\section{Solution}
Total number of cards =52 with 4 aces,48 non-ace's and we need to select 2 cards
so X can be 0 ,1 or 2\\ 

Let $A \in \{0,1\}$ represent the random variable, where 0 represents first card being an non ace, 1 represents first card being ace. \\
Let $B \in \{0,1\}$ represent the random variable, where 0 represents second card being an non-ace, 1 represents second card being ace \\ \\
\resizebox{8.5cm}{!}{
\begin{center}
\begin{tabular}{|c|c|c|c|}
\hline
{\pr{A=0}}& 48/52 &{\pr{A=1}}& 4/52  \\
\hline
\pr{B=0|A=0}&  47/51 &\pr{B=0|A=1}& 48/51 \\
\hline
{\pr{B=1|A=0}}& 4/51 &\pr{B=1|A=1}& 3/51  \\ 
\hline 
\end{tabular}
\end{center}
}
\hfill \\ \\
if A=1 then 3 aces left and if A=0 then\\ 4 aces left in remaining 51 cards\\ \\ 
Case 1: \emph{X} = 0
\begin{align}
\nonumber
&\implies \pr{X=0}=\pr{A=0,B=0}\\ \nonumber
&=\pr{A=0}\times \pr{B=0|A=0}\\ \nonumber
&\pr{X=0} =188/221\\
\end{align}

Case 2: \emph{X} = 1
\begin{align}
\nonumber
&\pr{X=1}=\pr{A=1,B=0}+\pr{A=0,B=1}\\ \nonumber 
&\pr{A=1,B=0}=\pr{A=1}\times \pr{B=0|A=1}\\ \nonumber
&\pr{A=1,B=0} =16/221\\ \nonumber
&\pr{A=0,B=1}=\pr{A=0}\times \pr{B=1|A=0}\\ \nonumber
&\pr{A=0,B=1} =16/221\\ \nonumber
&\implies\pr{X=1}=\frac{32}{221}\\
\end{align}
Case 3: \emph{X} = 2
\begin{align}
\nonumber
&\implies \pr{X=2}=\pr{A=1,B=1}\\ \nonumber
& =\pr{A=1}\times\pr{B=1|A=1}\\ \nonumber
& \pr{X=2}=1/221\\
\end{align}

 Now we know that E(X) denotes the average or expectation value which means that E(X) is the weighted average of all values X can take,each value being weighted by the probability of that particular event/value of X occurring\\  
 i.e E(X) is given by\[E(X) = {\sum_{i=0}^2 X\times \pr{X} }\]

\begin{center}
\begin{tabular}{|c|c|c|c|}
\hline
{\emph{X}} & 0 & 1 & 2  \\
\hline
{\pr{X}} &  188/221 &  32/221 &  1/221 \\
\hline
{\emph{X}$\times$ \pr{X}} & 0 & 32/221 & 2/221  \\
\hline 
\end{tabular}
\end{center}
$\implies E(X) = \frac{32 +2}{221} =\frac{2}{13}\\$

Final answer E(x) = 2/13 or option 4
\end{document}s
