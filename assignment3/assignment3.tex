\documentclass[journal,12pt,twocolumn]{IEEEtran}

\usepackage{setspace}
\usepackage{gensymb}
\singlespacing
\usepackage[cmex10]{amsmath}

\usepackage{amsthm}

\usepackage{mathrsfs}
\usepackage{txfonts}
\usepackage{stfloats}
\usepackage{bm}
\usepackage{cite}
\usepackage{cases}
\usepackage{subfig}

\usepackage{longtable}
\usepackage{multirow}

\usepackage{enumitem}
\usepackage{mathtools}
\usepackage{steinmetz}
\usepackage{tikz}
\usepackage{circuitikz}
\usepackage{verbatim}
\usepackage{tfrupee}
\usepackage[breaklinks=true]{hyperref}
\usepackage{graphicx}
\usepackage{tkz-euclide}

\usetikzlibrary{calc,math}
\usepackage{listings}
    \usepackage{color}                                            %%
    \usepackage{array}                                            %%
    \usepackage{longtable}                                        %%
    \usepackage{calc}                                             %%
    \usepackage{multirow}                                         %%
    \usepackage{hhline}                                           %%
    \usepackage{ifthen}                                           %%
    \usepackage{lscape}     
\usepackage{multicol}
\usepackage{chngcntr}

\DeclareMathOperator*{\Res}{Res}

\renewcommand\thesection{\arabic{section}}
\renewcommand\thesubsection{\thesection.\arabic{subsection}}
\renewcommand\thesubsubsection{\thesubsection.\arabic{subsubsection}}

\renewcommand\thesectiondis{\arabic{section}}
\renewcommand\thesubsectiondis{\thesectiondis.\arabic{subsection}}
\renewcommand\thesubsubsectiondis{\thesubsectiondis.\arabic{subsubsection}}


\hyphenation{op-tical net-works semi-conduc-tor}
\def\inputGnumericTable{}                                 %%

\lstset{
%language=C,
frame=single, 
breaklines=true,
columns=fullflexible
}
\begin{document}


\newtheorem{theorem}{Theorem}[section]
\newtheorem{problem}{Problem}
\newtheorem{proposition}{Proposition}[section]
\newtheorem{lemma}{Lemma}[section]
\newtheorem{corollary}[theorem]{Corollary}
\newtheorem{example}{Example}[section]
\newtheorem{definition}[problem]{Definition}

\newcommand{\BEQA}{\begin{eqnarray}}
\newcommand{\EEQA}{\end{eqnarray}}
\newcommand{\define}{\stackrel{\triangle}{=}}
\bibliographystyle{IEEEtran}
\raggedbottom
\setlength{\parindent}{0pt}
\providecommand{\mbf}{\mathbf}
\providecommand{\pr}[1]{\ensuremath{\Pr\left(#1\right)}}
\providecommand{\qfunc}[1]{\ensuremath{Q\left(#1\right)}}
\providecommand{\sbrak}[1]{\ensuremath{{}\left[#1\right]}}
\providecommand{\lsbrak}[1]{\ensuremath{{}\left[#1\right.}}
\providecommand{\rsbrak}[1]{\ensuremath{{}\left.#1\right]}}
\providecommand{\brak}[1]{\ensuremath{\left(#1\right)}}
\providecommand{\lbrak}[1]{\ensuremath{\left(#1\right.}}
\providecommand{\rbrak}[1]{\ensuremath{\left.#1\right)}}
\providecommand{\cbrak}[1]{\ensuremath{\left\{#1\right\}}}
\providecommand{\lcbrak}[1]{\ensuremath{\left\{#1\right.}}
\providecommand{\rcbrak}[1]{\ensuremath{\left.#1\right\}}}
\theoremstyle{remark}
\newtheorem{rem}{Remark}
\newcommand{\sgn}{\mathop{\mathrm{sgn}}}
\providecommand{\abs}[1]{$\left\vert#1\right\vert$}
\providecommand{\res}[1]{\Res\displaylimits_{#1}} 
\providecommand{\norm}[1]{$\left\lVert#1\right\rVert$}
%\providecommand{\norm}[1]{\lVert#1\rVert}
\providecommand{\mtx}[1]{\mathbf{#1}}
\providecommand{\mean}[1]{$\left[ #1 \right]$}

\providecommand{\fourier}{\overset{\mathcal{F}}{ \rightleftharpoons}}
%\providecommand{\hilbert}{\overset{\mathcal{H}}{ \rightleftharpoons}}
\providecommand{\system}{\overset{\mathcal{H}}{ \longleftrightarrow}}
	%\newcommand{\solution}[2]{\textbf{Solution:}{#1}}
\newcommand{\solution}{\noindent \textbf{Solution: }}
\newcommand{\cosec}{\,\text{cosec}\,}
\providecommand{\dec}[2]{\ensuremath{\overset{#1}{\underset{#2}{\gtrless}}}}
\newcommand{\myvec}[1]{\ensuremath{\begin{pmatrix}#1\end{pmatrix}}}
\newcommand{\mydet}[1]{\ensuremath{\begin{vmatrix}#1\end{vmatrix}}}
\numberwithin{equation}{subsection}
\makeatletter
\@addtoreset{figure}{problem}
\makeatother
\let\StandardTheFigure\thefigure
\let\vec\mathbf
\renewcommand{\thefigure}{\theproblem}
\def\putbox#1#2#3{\makebox[0in][l]{\makebox[#1][l]{}\raisebox{\baselineskip}[0in][0in]{\raisebox{#2}[0in][0in]{#3}}}}
     \def\rightbox#1{\makebox[0in][r]{#1}}
     \def\centbox#1{\makebox[0in]{#1}}
     \def\topbox#1{\raisebox{-\baselineskip}[0in][0in]{#1}}
     \def\midbox#1{\raisebox{-0.5\baselineskip}[0in][0in]{#1}}
\vspace{3cm}
\title{Assignment 3}
\author{Ojjas Tyagi - CS20BTECH11060}
\maketitle
\newpage
\bigskip
\renewcommand{\thefigure}{\theenumi}
\renewcommand{\thetable}{\theenumi}
Download all python codes from 
\begin{lstlisting}
https://github.com/tyagio/AI1103/tree/main/assignment3/codes
\end{lstlisting}
%
and latex-tikz codes from 
%
\begin{lstlisting}
https://github.com/tyagio/AI1103/tree/main/assignment3/assignment3.tex
\end{lstlisting}
\section{Problem}
Let X1,X2... be a sequence of independent and identically distributed random variable with
\begin{align}
    \pr{X_1=-1}=\pr{X_1=1}=1/2
\end{align}
Suppose for the standard normal random variable Z,\pr{-0.1\leq Z \leq 0.1}=0.08 .\\
If \(S_n= \sum_{i=1}^{n^2} X_i ,\text{then} \lim_{x\to\infty} \pr{S_n >\frac{n}{10}}\) =
\begin{enumerate}
    \item 0.42
    \item 0.46
    \item 0.50
    \item 0.54
\end{enumerate}
\section{Solution}
\[p_{X_i} (n)=\pr{X_i = n}= \begin{cases}
            \frac{1}{2}, &\text{if n= 1 or n=-1}\\
             0, &\text{otherwise}\\
            \end{cases}
\]
the Z transform for \(X_i\) is given by\\
\begin{align}
&P_{X_i} (z)=\frac{1}{2}(z + z^{-1})\\
\implies& P_{S_n} (z)= \prod_{i=1}^{n^2} P_{X_i}(z)\\
\implies& P_{S_n} (z)= \frac{1}{2^{n^2}}\sum_{i=1}^{n^2} {{n^2}\choose{i}} z^{{n^2}-i} z^{-i}
\end{align}
Now we can say that to find the the probability that \(-n/10 \leq S_n  \leq n/10\) is the same as finding the sum of all terms in \(P_{S_n}(z)\) where coefficient of z is between -n/10 and n/10,WOLG we can say 
\begin{align}
    \pr{\frac{-n}{10} \leq S_n \leq \frac{n}{10}}= \frac{1}{2^{n^2}}\sum_{i={n^2/2-n/20}}^{n^2/2 +n/20} {n^2\choose i}
\end{align}
Using the approximation
\begin{align}
    &{n\choose k}{p^k}{q^{n-k}}\simeq \frac{1}{\sqrt{ 2\pi npq}} \exp(-(k-np)^2)/(2npq))  \\
    &\text{In our case } p=q=1/2 \nonumber
\end{align}
This holds for large n and hence as our n tends to infinity we can transform the summation into an integral
which is given by
\begin{align}
   &\frac{\sqrt{2}}{\sqrt{\pi{n^2}}} \int_{{n^2}/2 -n/20}^{{n^2}/2 +n/20}
   \exp(-(k-\frac{n^2}{2})^2)/(\frac{n^2}{2})) dk\\
   \implies &\frac{\sqrt{2}}{n\sqrt{\pi}} \int_{-n/20}^{n/20}  \exp(-2 k^2/n^2) dk\\
   \implies &\frac{1}{\sqrt{2\pi}} \int_{-1/10}^{1/10}  \exp(-k^2/2) dk
\end{align}
This is just \(G(0.1) - G(-0.1)\) which is given to us in the question as 0.08
\begin{align}
\implies \pr{\frac{-n}{10} \leq S_n \leq \frac{n}{10}}=0.08
\end{align}
(variable Z corresponds to standard normal random variable or \( N(0,1)\).)\\ 
The probability distribution of \(S_n\) is symmetric about 0,hence \pr{S_n>\frac{n}{10}} = \pr{S_n<\frac{-n}{10}}\\
\begin{align}
  &  \implies 2\times \pr{S_n>\frac{n}{10}} + \pr{\frac{-n}{10} \leq S_n \leq \frac{n}{10}} =1\\
  &  \implies  \pr{S_n>\frac{n}{10}}=0.92/2=0.46
\end{align}
Hence final solution is options 2) or 0.46
\end{document}