\documentclass{beamer}
\usepackage{listings}
\lstset{
%language=C,
frame=single, 
breaklines=true,
columns=fullflexible
}
\usepackage{subcaption}
\usepackage{url}

\usepackage{tikz}
\usepackage{pgfplots}
\pgfplotsset{compat=1.17}
\usepackage{tkz-fct}
\usepackage{mathrsfs}
\usepackage{txfonts}
\usepackage{tkz-euclide} 
\usetikzlibrary{calc,math}
\usepackage{float}
\newcommand\norm[1]{\left\lVert#1\right\rVert}
\renewcommand{\vec}[1]{\mathbf{#1}}
\providecommand{\pr}[1]{\ensuremath{\Pr\left(#1\right)}}
\usepackage[export]{adjustbox}
\usepackage[utf8]{inputenc}
\usepackage{amsmath}
\usetheme{Boadilla}
\title{CSIR UGC NET EXAM (Dec 2018), Q.49}
\author{Ojjas Tyagi - MA20BTECH1102/CS20BTECH11060}
\begin{document}
\begin{frame}
\titlepage
\end{frame}
\section{Question}
\begin{frame}
\frametitle{CSIR UGC NET EXAM (Dec 2018), Q.49}
\begin{block}{Question}
Let \(X \geq 0\) be a random variable on \(  (\Omega,\mathcal{F},P) \) with E(\emph{X})= 1 .Let A \(\in \mathcal{F}\) be an event with \(0<P(A)<1\). Which of the following defines another probability measure on \((\Omega,\mathcal{F})\) ?
\begin{enumerate}
    \item \(Q(B) = P(A\cap B)\quad \forall B \in \mathcal{F} \)
    \item \(Q(B) = P(A\cup B)\quad \forall B \in \mathcal{F} \)
    \item \(Q(B) = E(XI_B) \quad\,\,\,\,\, \forall B \in \mathcal{F} \)
    \item Q(B) =\( \begin{cases}
         P(A|B), &\text{if \(P(B)>0\)}\\
         0, &\text{if \(P(B)=0\)}\\
   \end{cases} \)
\end{enumerate}
\end{block}
\end{frame}
\section{Definitions}
\subsection{Sample space}
\begin{frame}{Sample space \(\Omega\)}
\begin{block}{Definition}
Set  of all possible outcomes of a random experiment
\end{block}
\end{frame}
\subsection{Event space}
\begin{frame}{Event space \(\mathcal{F}\)}
\begin{block}{Events}
Subsets of \(\Omega\) which are of interest
\end{block}
\begin{block}{\(\sigma\)-algebra}
A collection \(\mathcal{F}\) of subsets of \(\Omega\) is called a \(\sigma\)-algebra if:
\begin{enumerate}
    \item \(\phi \in \mathcal{F}\) 
    \item  if \( A \in \mathcal{F}\) then \(A^C \in \mathcal{F}\)
    \item if \(A_1,A_2,....\) is a countable collection of subsets in \(\mathcal{F}\),then
    \(\bigcup_{i=1}^{\infty} A_i \in \mathcal{F}\)
Last point also allows us to state that(Using De Morgan's law):\\
if \(A_1,A_2,....\) is a countable collection of subsets in \(\mathcal{F}\),then
    \(\bigcap_{i=1}^{\infty} A_i \in \mathcal{F}\)\\
Subsets in \(\mathcal{F}\) called \(\mathcal{F}\)-measurable sets   
\end{enumerate}
\end{block}
\begin{block}{Measurable space}
     \((\Omega,\mathcal{F})\) is known as a measurable space
\end{block}
\end{frame}
\subsection{Probability measure}
\begin{frame}{Measure}
\begin{block}{Definition}
A measure is a function \(\mu:\mathcal{F}\to[0,\infty)\) such that:
\begin{enumerate}
    \item \(\mu(\phi)\)=0
    \item if \(A_1,A_2,....\) is a countable collection of disjoint \(\mathcal{F}\)-measurable sets, then
    \(\mu(\bigcup_{i=1}^{\infty} A_i)=\sum_{i=1}^{\infty} \mu(A_i)\)
\end{enumerate}
\end{block}
\begin{block}{Measure space}
The triple (\(\Omega,\mathcal{F},\mu\)) is called a measure space.\\
if \(\mu(\Omega)<\infty\) then \(\mu\) is called a finite measure.\\
\end{block}
\begin{block}{Probability measure}
 A probability measure \(\mathbb{P}\) on\((\Omega,\mathcal{F})\) is a measure with the special property:
 \[\mathbb{P}(\Omega)=1\]
\end{block}
\end{frame}
\subsection{random variables borel sets}
\begin{frame}{Random variables \& Borel sets}
\begin{block}{Borel sets}
      A Borel set is any set that can be formed from open sets through the operations of countable union, countable intersection, and complement on some set X.\\
      Alternatively a borel set is a member of the borel \(\sigma\) algebra which is the smallest \(\sigma\)-algebra containing all open sets on some set X and is denoted by \(\mathcal{B}(X)\).\\
      i.e if \(B \in \mathcal{B}(X)\) then B is a borel set in X.
\end{block}
\end{frame}
\begin{frame}
\begin{block}{\(\mathcal{F}\)-measurable functions}
   A function \(X:\Omega \to \mathbb{R}\) is said to be \(\mathcal{F}\)-measurable if for every
   \(B \in \mathcal{B}(\mathbb{R})\),the pre-image \(X^{-1}(B) \in \mathcal{F}\)\\
   Where \(X^{-1}(B)=\{\omega \in \Omega|X(\omega) \in B\}\)
\end{block}
\begin{block}{Random variable}
     A random variable X on the probability space \((\Omega,\mathcal{F},\mathbb{P})\) is a \(\mathcal{F}\)-measurable function \(X:\Omega \to \mathbb{R}\)
\end{block}
\end{frame}
\subsection{Solution}
\begin{frame}{Solution}
\begin{block}{Probability measure}
  Any probability measure \(\mathbb{P}\) on \((\Omega,\mathcal{F})\) is a function from \(\mathcal{F}\) to [0,1] with the following properties
\begin{enumerate}
   \item \(\mathbb{P}(\phi)=0 \)
   \item \(\mathbb{P}(\Omega)=1 \)
   \item \text{If \(A_1,A_2\).... are disjoint \(\mathcal{F}\)-measurable sets}\\
    \(\text{then } \mathbb{P}(\bigcup_{i=1}^{\infty} A_i)= \sum_{i=1}^{\infty} \mathbb{P}(A_i) \)
\end{enumerate}
\end{block}
\end{frame}
\begin{frame}
\begin{block}{Conditions}
Now we know that \(E(XI_B)\geq 0\, \forall\, B\text{ as } X\geq 0\)\\
also that for \(Q(B) =E(XI_B)\quad  \forall B \in \mathcal{F}\) to be valid probability measure
\begin{align}
& E(XI_B)=E(X\times 1)=1  \text{ when } B=\Omega \label{eqn1}\\ 
& E(XI_B)=E(X\times0)=0  \text{ when } B=\phi  \label{eqn2} \\ \nonumber
 \text{Let } &B'=\bigcup_{i=1}^{\infty} A_i \text{ ,where } A_i\text{ are disjoint sets}\in \mathcal{F}\\ 
\implies & E(XI_{B'})=\sum_{i=1}^{\infty} E(XI_{A_i}) \label{eqn3}
\end{align}
\end{block}
\end{frame}
\begin{frame}
\begin{block}{Proof}
Now to prove that \eqref{eqn3} is true
\begin{align}
& E(XI_B)=\sum_{\omega \in \Omega}{ X(\omega)P(\omega)\times I_B(\omega)}\\ \nonumber
&\text{Now \(I_B(\omega)\) is indicator function which}\\
&\text{is 1 if }\omega \in B \text{ and 0 if } \omega \notin  B\\
\implies & E(XI_B)=\sum_{\omega \in B} {X(\omega)P(\omega)}\\
\implies & \sum_{i=1}^{\infty} E(XI_{A_i})=\sum_{i=1}^{\infty} \sum_{\omega \in A_i}{X(\omega)P(\omega)}\\
\implies & \sum_{\omega \in B'}{X(\omega)P(\omega)}=E(XI_B')
\end{align} 
This proves that \eqref{eqn3} is true
\end{block}
\end{frame}
\begin{frame}
\begin{block}{}
Similarly to prove \eqref{eqn1} and \eqref{eqn2}
\begin{align} \nonumber
&\text{if }B =\phi \\
&\implies E(XI_B)=\sum_{\omega \in \Omega}{ X(\omega)P(\omega)\times 0}=0\\ \nonumber
&\text{Also if }B=\Omega\\
&\implies E(XI_B)=\sum_{\omega \in \Omega}{ X(\omega) P(\omega)\times 1}=E(X)=1
\end{align}
This proves that \eqref{eqn1} and \eqref{eqn2} are true
\end{block}
\end{frame}
\begin{frame}{Answer}
\begin{block}{Correct answer}
Now using \eqref{eqn1},\eqref{eqn2} and \eqref{eqn3} we can state that \[Q(B)=E(XI_B)\quad\forall B \in \mathcal{F}\] is a well defined probability measure.
\(\implies\) option 3 is correct answer.  
\end{block}
\end{frame}
\end{document}