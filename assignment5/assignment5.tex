\documentclass[journal,12pt,twocolumn]{IEEEtran}

\usepackage{setspace}
\usepackage{gensymb}
\singlespacing
\usepackage[cmex10]{amsmath}

\usepackage{amsthm}

\usepackage{mathrsfs}
\usepackage{txfonts}
\usepackage{stfloats}
\usepackage{bm}
\usepackage{cite}
\usepackage{cases}
\usepackage{subfig}

\usepackage{longtable}
\usepackage{multirow}

\usepackage{enumitem}
\usepackage{mathtools}
\usepackage{steinmetz}
\usepackage{tikz}
\usepackage{circuitikz}
\usepackage{verbatim}
\usepackage{tfrupee}
\usepackage[breaklinks=true]{hyperref}
\usepackage{graphicx}
\usepackage{tkz-euclide}

\usetikzlibrary{calc,math}
\usepackage{listings}
    \usepackage{color}                                            %%
    \usepackage{array}                                            %%
    \usepackage{longtable}                                        %%
    \usepackage{calc}                                             %%
    \usepackage{multirow}                                         %%
    \usepackage{hhline}                                           %%
    \usepackage{ifthen}                                           %%
    \usepackage{lscape}     
\usepackage{multicol}
\usepackage{chngcntr}

\DeclareMathOperator*{\Res}{Res}

\renewcommand\thesection{\arabic{section}}
\renewcommand\thesubsection{\thesection.\arabic{subsection}}
\renewcommand\thesubsubsection{\thesubsection.\arabic{subsubsection}}

\renewcommand\thesectiondis{\arabic{section}}
\renewcommand\thesubsectiondis{\thesectiondis.\arabic{subsection}}
\renewcommand\thesubsubsectiondis{\thesubsectiondis.\arabic{subsubsection}}


\hyphenation{op-tical net-works semi-conduc-tor}
\def\inputGnumericTable{}                                 %%

\lstset{
%language=C,
frame=single, 
breaklines=true,
columns=fullflexible
}
\begin{document}


\newtheorem{theorem}{Theorem}[section]
\newtheorem{problem}{Problem}
\newtheorem{proposition}{Proposition}[section]
\newtheorem{lemma}{Lemma}[section]
\newtheorem{corollary}[theorem]{Corollary}
\newtheorem{example}{Example}[section]
\newtheorem{definition}[problem]{Definition}

\newcommand{\BEQA}{\begin{eqnarray}}
\newcommand{\EEQA}{\end{eqnarray}}
\newcommand{\define}{\stackrel{\triangle}{=}}
\bibliographystyle{IEEEtran}
\raggedbottom
\setlength{\parindent}{0pt}
\providecommand{\mbf}{\mathbf}
\providecommand{\pr}[1]{\ensuremath{\Pr\left(#1\right)}}
\providecommand{\qfunc}[1]{\ensuremath{Q\left(#1\right)}}
\providecommand{\sbrak}[1]{\ensuremath{{}\left[#1\right]}}
\providecommand{\lsbrak}[1]{\ensuremath{{}\left[#1\right.}}
\providecommand{\rsbrak}[1]{\ensuremath{{}\left.#1\right]}}
\providecommand{\brak}[1]{\ensuremath{\left(#1\right)}}
\providecommand{\lbrak}[1]{\ensuremath{\left(#1\right.}}
\providecommand{\rbrak}[1]{\ensuremath{\left.#1\right)}}
\providecommand{\cbrak}[1]{\ensuremath{\left\{#1\right\}}}
\providecommand{\lcbrak}[1]{\ensuremath{\left\{#1\right.}}
\providecommand{\rcbrak}[1]{\ensuremath{\left.#1\right\}}}
\theoremstyle{remark}
\newtheorem{rem}{Remark}
\newcommand{\sgn}{\mathop{\mathrm{sgn}}}
\providecommand{\abs}[1]{$\left\vert#1\right\vert$}
\providecommand{\res}[1]{\Res\displaylimits_{#1}} 
\providecommand{\norm}[1]{$\left\lVert#1\right\rVert$}
%\providecommand{\norm}[1]{\lVert#1\rVert}
\providecommand{\mtx}[1]{\mathbf{#1}}
\providecommand{\mean}[1]{$\left[ #1 \right]$}

\providecommand{\fourier}{\overset{\mathcal{F}}{ \rightleftharpoons}}
%\providecommand{\hilbert}{\overset{\mathcal{H}}{ \rightleftharpoons}}
\providecommand{\system}{\overset{\mathcal{H}}{ \longleftrightarrow}}
	%\newcommand{\solution}[2]{\textbf{Solution:}{#1}}
\newcommand{\solution}{\noindent \textbf{Solution: }}
\newcommand{\cosec}{\,\text{cosec}\,}
\providecommand{\dec}[2]{\ensuremath{\overset{#1}{\underset{#2}{\gtrless}}}}
\newcommand{\myvec}[1]{\ensuremath{\begin{pmatrix}#1\end{pmatrix}}}
\newcommand{\mydet}[1]{\ensuremath{\begin{vmatrix}#1\end{vmatrix}}}
\numberwithin{equation}{subsection}
\makeatletter
\@addtoreset{figure}{problem}
\makeatother
\let\StandardTheFigure\thefigure
\let\vec\mathbf
\renewcommand{\thefigure}{\theproblem}
\def\putbox#1#2#3{\makebox[0in][l]{\makebox[#1][l]{}\raisebox{\baselineskip}[0in][0in]{\raisebox{#2}[0in][0in]{#3}}}}
     \def\rightbox#1{\makebox[0in][r]{#1}}
     \def\centbox#1{\makebox[0in]{#1}}
     \def\topbox#1{\raisebox{-\baselineskip}[0in][0in]{#1}}
     \def\midbox#1{\raisebox{-0.5\baselineskip}[0in][0in]{#1}}
\vspace{3cm}
\title{Assignment 5}
\author{Ojjas Tyagi - CS20BTECH11060}
\maketitle
\newpage
\bigskip
\renewcommand{\thefigure}{\theenumi}
\renewcommand{\thetable}{\theenumi}
Download  latex-tikz codes from 
%
\begin{lstlisting}
https://github.com/tyagio/AI1103/tree/main/assignment5/assignment5.tex
\end{lstlisting}
\section{Problem}
Let \(X \geq 0\) be a random variable on \(  (\Omega,\mathcal{F},P) \) with E(\emph{X})= 1 .Let A \(\in \mathcal{F}\) be an event with \(0<P(A)<1\). Which of the following defines another probability measure on \((\Omega,\mathcal{F})\) ?
\begin{enumerate}
    \item \(Q(B) = P(A\cap B)\quad \forall B \in \mathcal{F} \)
    \item \(Q(B) = P(A\cup B)\quad \forall B \in \mathcal{F} \)
    \item \(Q(B) = E(XI_B) \quad\,\,\,\,\, \forall B \in \mathcal{F} \)
    \item Q(B) =\( \begin{cases}
         P(A|B), &\text{if \(P(B)>0\)}\\
         0, &\text{if \(P(B)=0\)}\\
   \end{cases} \)
\end{enumerate}
\section{Solution}
Any probability measure \(\mathbb{P}\) on \((\Omega,\mathcal{F})\) is a function from \(\mathcal{F}\) to [0,1] with the following properties
\begin{enumerate}
   \item \(\mathbb{P}(\phi)=0 \)
   \item \(\mathbb{P}(\Omega)=1 \)
   \item \(\text{If \(A_1,A_2\).... are disjoint \(\mathcal{F}\)-measurable sets}\\
    \text{then } \mathbb{P}\brak{\bigcup_{i=1}^{\infty} A_i}= \sum_{i=1}^{\infty} \mathbb{P}(A_i)\\ \)
\end{enumerate}
Now we know that \(E(XI_B)\geq 0\, \forall\, B\text{ as } X\geq 0\)\\
also 
\begin{align}
& E(XI_B)=E(X\times 1)=1  \text{ when } B=\Omega \label{eqn1}\\ 
& E(XI_B)=E(X\times0)=0  \text{ when } B=\phi  \label{eqn2} \\ \nonumber
& \text{Let } B=\bigcup_{i=1}^{\infty} A_i \text{ ,where } A_i\text{ are disjoint sets}\in \mathcal{F}\\ 
\implies & E(XI_B)=\sum_{i=1}^{\infty} E(XI_{A_i}) \label{eqn3}
\end{align}
\begin{enumerate}
\item \eqref{eqn3} is true because Indicator function \(I_B\) is 1 when element within B and 0 otherwise due to this the summation in the RHS is equivalent to \( E(XI_B)\) but calculated for smaller intervals where X is set to 0 when it is not in disjoint subinterval of B and then summed up over all subintervals to get same value as LHS.
\item similarly \eqref{eqn1} and \eqref{eqn2} are true as \(I_B\) is always 0 for any \(\omega \in \Omega\) when \(B=\phi\) and \(I_B\) is always 1 for any \(\omega \in \Omega\) when \(B =\Omega\)
\end{enumerate} 
Now we can state that \[Q(B)=E(XI_B)\quad\forall B \in \mathcal{F}\] is a well defined probability measure.

\(\implies\) option 3 is correct answer.
\end{document}
