\documentclass[journal,12pt,twocolumn]{IEEEtran}

\usepackage{setspace}
\usepackage{gensymb}
\singlespacing
\usepackage[cmex10]{amsmath}

\usepackage{amsthm}

\usepackage{mathrsfs}
\usepackage{txfonts}
\usepackage{stfloats}
\usepackage{bm}
\usepackage{cite}
\usepackage{cases}
\usepackage{subfig}

\usepackage{longtable}
\usepackage{multirow}

\usepackage{enumitem}
\usepackage{mathtools}
\usepackage{steinmetz}
\usepackage{tikz}
\usepackage{circuitikz}
\usepackage{verbatim}
\usepackage{tfrupee}
\usepackage[breaklinks=true]{hyperref}
\usepackage{graphicx}
\usepackage{tkz-euclide}

\usetikzlibrary{calc,math}
\usepackage{listings}
    \usepackage{color}                                            %%
    \usepackage{array}                                            %%
    \usepackage{longtable}                                        %%
    \usepackage{calc}                                             %%
    \usepackage{multirow}                                         %%
    \usepackage{hhline}                                           %%
    \usepackage{ifthen}                                           %%
    \usepackage{lscape}     
\usepackage{multicol}
\usepackage{chngcntr}

\DeclareMathOperator*{\Res}{Res}

\renewcommand\thesection{\arabic{section}}
\renewcommand\thesubsection{\thesection.\arabic{subsection}}
\renewcommand\thesubsubsection{\thesubsection.\arabic{subsubsection}}

\renewcommand\thesectiondis{\arabic{section}}
\renewcommand\thesubsectiondis{\thesectiondis.\arabic{subsection}}
\renewcommand\thesubsubsectiondis{\thesubsectiondis.\arabic{subsubsection}}


\hyphenation{op-tical net-works semi-conduc-tor}
\def\inputGnumericTable{}                                 %%

\lstset{
%language=C,
frame=single, 
breaklines=true,
columns=fullflexible
}
\begin{document}


\newtheorem{theorem}{Theorem}[section]
\newtheorem{problem}{Problem}
\newtheorem{proposition}{Proposition}[section]
\newtheorem{lemma}{Lemma}[section]
\newtheorem{corollary}[theorem]{Corollary}
\newtheorem{example}{Example}[section]
\newtheorem{definition}[problem]{Definition}

\newcommand{\BEQA}{\begin{eqnarray}}
\newcommand{\EEQA}{\end{eqnarray}}
\newcommand{\define}{\stackrel{\triangle}{=}}
\bibliographystyle{IEEEtran}
\raggedbottom
\setlength{\parindent}{0pt}
\providecommand{\mbf}{\mathbf}
\providecommand{\pr}[1]{\ensuremath{\Pr\left(#1\right)}}
\providecommand{\qfunc}[1]{\ensuremath{Q\left(#1\right)}}
\providecommand{\sbrak}[1]{\ensuremath{{}\left[#1\right]}}
\providecommand{\lsbrak}[1]{\ensuremath{{}\left[#1\right.}}
\providecommand{\rsbrak}[1]{\ensuremath{{}\left.#1\right]}}
\providecommand{\brak}[1]{\ensuremath{\left(#1\right)}}
\providecommand{\lbrak}[1]{\ensuremath{\left(#1\right.}}
\providecommand{\rbrak}[1]{\ensuremath{\left.#1\right)}}
\providecommand{\cbrak}[1]{\ensuremath{\left\{#1\right\}}}
\providecommand{\lcbrak}[1]{\ensuremath{\left\{#1\right.}}
\providecommand{\rcbrak}[1]{\ensuremath{\left.#1\right\}}}
\theoremstyle{remark}
\newtheorem{rem}{Remark}
\newcommand{\sgn}{\mathop{\mathrm{sgn}}}
\providecommand{\abs}[1]{$\left\vert#1\right\vert$}
\providecommand{\res}[1]{\Res\displaylimits_{#1}} 
\providecommand{\norm}[1]{$\left\lVert#1\right\rVert$}
%\providecommand{\norm}[1]{\lVert#1\rVert}
\providecommand{\mtx}[1]{\mathbf{#1}}
\providecommand{\mean}[1]{$\left[ #1 \right]$}

\providecommand{\fourier}{\overset{\mathcal{F}}{ \rightleftharpoons}}
%\providecommand{\hilbert}{\overset{\mathcal{H}}{ \rightleftharpoons}}
\providecommand{\system}{\overset{\mathcal{H}}{ \longleftrightarrow}}
	%\newcommand{\solution}[2]{\textbf{Solution:}{#1}}
\newcommand{\solution}{\noindent \textbf{Solution: }}
\newcommand{\cosec}{\,\text{cosec}\,}
\providecommand{\dec}[2]{\ensuremath{\overset{#1}{\underset{#2}{\gtrless}}}}
\newcommand{\myvec}[1]{\ensuremath{\begin{pmatrix}#1\end{pmatrix}}}
\newcommand{\mydet}[1]{\ensuremath{\begin{vmatrix}#1\end{vmatrix}}}
\numberwithin{equation}{subsection}
\makeatletter
\@addtoreset{figure}{problem}
\makeatother
\let\StandardTheFigure\thefigure
\let\vec\mathbf
\renewcommand{\thefigure}{\theproblem}
\def\putbox#1#2#3{\makebox[0in][l]{\makebox[#1][l]{}\raisebox{\baselineskip}[0in][0in]{\raisebox{#2}[0in][0in]{#3}}}}
     \def\rightbox#1{\makebox[0in][r]{#1}}
     \def\centbox#1{\makebox[0in]{#1}}
     \def\topbox#1{\raisebox{-\baselineskip}[0in][0in]{#1}}
     \def\midbox#1{\raisebox{-0.5\baselineskip}[0in][0in]{#1}}
\vspace{3cm}
\title{Assignment 6}
\author{Ojjas Tyagi - CS20BTECH11060}
\maketitle
\newpage
\bigskip
\renewcommand{\thefigure}{\theenumi}
\renewcommand{\thetable}{\theenumi}
Download all python codes from 
\begin{lstlisting}
https://github.com/tyagio/AI1103/tree/main/assignment6/codes
\end{lstlisting}
%
and latex-tikz codes from 
%
\begin{lstlisting}
https://github.com/tyagio/AI1103/tree/main/assignment6/assignment6.tex
\end{lstlisting}
\section{Problem}
Let X and Y be i.i.d random variables uniformly distributed on (0,4).Then \pr{X>Y|X<2Y} is
\begin{enumerate}
    \item 1/3
    \item 5/6
    \item 1/4
    \item 2/3
\end{enumerate}
\section{Solution}
The PDF is given by
\begin{align}
   &f_X (x)=f_Y (x)=\nonumber \begin{cases}
         \frac{1}{4}, &\text{if 0 \(< x <\) 4}\\
         0, &\text{otherwise}\\
   \end{cases} 
\end{align}    
The CDF is given by
\begin{align}
   \nonumber& F(x)=\int_{-\infty}^{x} f(x)dx \\ \nonumber
   &F_X (x)=F_Y (x)=\nonumber \begin{cases}
          0, & x\leq 0\\
         \frac{x}{4}, &\text{if 0 \(< x <\) 4}\\
          1, &x\geq4\\
   \end{cases}    
\end{align}
Using definition of conditional probability 
\begin{align}
    &\pr{X>Y|X<2Y}=\frac{\pr{Y < X< 2Y}}{\pr{X<2Y}} \label{eqn1}
\end{align}
Now finding \pr{X<2Y}
\begin{align}
    &\pr{X<2y}=F_X (2y)\\
    \implies& \pr{X<2Y}=\int_{-\infty}^{\infty} f_Y(x) \times F_X (2x)dx\\
    \implies& \pr{X<2Y}=\int_{0}^{2} \frac{x}{8}dx +\int_{2}^{4}\frac{1}{4}dx\\
    \implies& \pr{X<2Y}=\frac{3}{4}=0.75 \label{eqn2}
\end{align}
Now to find \pr{Y<X<2Y}
\begin{align}
    &\pr{y<X<2y}=F_X (2y)- F_X (y) \\
    \implies &\pr{Y<X<2Y}\\ \nonumber 
    &=\int_{-\infty}^{\infty} f_Y (x)( F_X (2x)- F_X(x))dx \\
   \implies &\int_{0}^{2}\frac{1}{4}\brak{\frac{x}{2}-\frac{x}{4}} dx +\int_{2}^{4}\frac{1}{4}\brak{1-\frac{x}{4}} dx\\
   \implies &\pr{Y<X<2Y}=\frac{1}{4}=0.25 \label{eqn3}
\end{align}
Now using \eqref{eqn1},\eqref{eqn2} and \eqref{eqn3}
\begin{align}
    \pr{X>Y|X<2Y}=\frac{1/4}{3/4}=\frac{1}{3}
\end{align}
Hence final solution is option 1) or 1/3 
\end{document}